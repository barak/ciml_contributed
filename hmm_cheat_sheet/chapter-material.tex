\chapter{Hidden Markov Model Cheat Sheet} \label{sec:hmm-cheat}

\chapterquote{The vampires of this world don't know the pleasures of
  hunger.  They gorge themselves without savoring the kill.}{Sorin
  Markov}

%\chapterimageopt{serious.png}{width=5cm}{Courtesy XKCD}

\begin{learningobjectives}
\item Understand what a Hidden Markov Model is.
\item Know the three important HMM problems and the algorithms for
  solving them.
\item The probability of a given sequence of observations.
\item The most probable sequence of hidden states for a given sequence
  of observations.
\item Update the HMM parameters to increase the probability of given
  sequences of observations.
\end{learningobjectives}

\dependencies{xxx.}

This document is a ``cheat sheet'' on Hidden Markov Models (HMMs).  It
resembles lecture notes, except that it cuts to the chase a little
faster by defining terms and divulging the useful formulas as quickly
as possible, in the place of gentle explanations and intuitions.

\section{Notation}

HMM:
\begin{compactitem}
\item states are not observable.
\item observations are probabilistic function of state
\item state transitions are probabilistic
\end{compactitem}

\begin{compactdesc}
\item[$N$:] number of hidden states, numbered $1,\ldots,N$
\item[$M$:] number of output symbols, numbered $1,\ldots,M$
\item[$T$:] number of time steps in sequence of states and sequence of
  output symbols
\item[$\vv{q}$:] sequence of states traversed, $\vv{q} = (q_1,
  \ldots, q_t, \ldots, q_T)$ where each $q_t \in \{1,\ldots,N\}$
\item[$\vv{o}$:] observed output symbol sequence, $\vv{o} = (o_1,
  \ldots, o_t, \ldots, o_T)$ where $o_t \in \{1,\ldots,M\}$
\item[$\mathbf{A}$:] state transition matrix,
  $a_{ij} = P(q_{t+1} = j \given q_t = i)$
\item[$B$:] per-state observation distributions,
  $b_i(k) = P(o_t = k \given q_t = i)$
\item[$\vv{\pi}$:] initial state distribution, $\pi_i = P(q_1 = i)$
\item[$\lam$:] all numeric parameters defining the HMM considered
  together, $\lam = (\mathbf{A}, B, \vv{\pi})$
\item[indices:] $i$, $j$ index states; $k$ indexes output symbols; $t$
  indexes time
\end{compactdesc}

We proceed to review the solutions to the three big HMM problems:
finding $P(\vv{o} \given \lam)$, finding $\vv{q}^* = \argmax_{\vv{q}}
P(\vv{q} \given \vv{o}, \lam)$, and finding $\lam^* = \argmax_{\lam}
P(\vv{o} \given \lam)$.

\section{Probability of sequence of observations}

We wish to calculate $P(\vv{o} \given \lam)$.

Definition: $\alpha_t(i) = P(o_1, \ldots,o_t,q_t=i \given \lam)$.
(In words: the probability of observing the head of length $t$ of the
observations and being in state $i$ after that.)

Initialization:
$\alpha_1(i) = \pi_i \, b_i(o_1)$.

Loop:
$\alpha_{t+1}(j)
= \biggl( \dsum_{i=1}^N \alpha_t(i) \, a_{ij} \biggr) \, b_j(o_{t+1})$

At termination, $P(\vv{o} \given \lam) = \dsum_{i=1}^N \alpha_T(i)$.

Note: complexity is $\OO(N^2 T)$ time, $\OO(N T)$ space.

Note: calculating the $\alpha$ values is called the ``forward algorithm.''

\section{Optimal state sequence from observations}

Find $\vv{q}^* = \argmax_{\vv{q}} P(\vv{q} \given \vv{o}, \lam)$,
the most likely sequence of hidden states given the observations.

Note: calculating the most likely sequence of states is called a
``Viterbi alignment.''

Definition: $\beta_t(i)
 = P(o_{t+1}, o_{t+2}, \ldots, o_T \given q_t = i, \lam)$.
(In words: the probability that starting in state $i$ at time $t$, then
generating the remaining tail of the observations.)

Initialization: $\beta_T(i) = 1$.

Loop: $\beta_t(i) = \dsum_{j=1}^N a_{ij} b_j(o_{t+1}) \beta_{t+1}(j)$.
Calculated backwards: $t = T-1, T-2, \ldots, 1$.

Note: calculating the $\beta$ values is called the ``backward algorithm.''

Define:
\begin{equation}
  \delta_t(i)
  = \max_{q_1, \ldots, q_{t-1}}
  P(q_1, \ldots, q_{t-1}, q_t=i, o_1, \ldots, o_t \given \lam) .
\end{equation}
(In words: the probability of generating the head of length $t$ of
 observables and having gone through the most likely states for
 the first $t-1$ steps and ending up in state $i$.)

Initialization: $\delta_1(i) = \pi_i \, b_i(o_1)$

Loop: $\delta_t(j) = (\displaystyle \max_i \delta_{t-1}(i) \, a_{ij}) \, b_j(o_t)$

Initialization: $\psi_1(i) = 0$

Loop: $\psi_t(j) = \displaystyle \argmax_i \delta_{t-1}(i) \, a_{ij}$

Termination: $P^* = \displaystyle \max_{i} \delta_T(i)$, the probability of
generating the entire sequence of observables via the most probable
sequence of states.

Termination: $q^*_T = \displaystyle \argmax_i \delta_T(i)$, the most probable final
state.

Loop to find state sequence (``backtracking''):
$q_t^* = \psi_{t+1}(q^*_{t+1})$

Note: $\psi$ is written ``psi'' in English, and pronounced ``p'sai.''

\subsection{Useful property of $\alpha$ and $\beta$}

Note that
\begin{equation*}
  \begin{split}
    \sum_i \alpha_t(i) \, \beta_t(i) &=
    \sum_i P(o_1, \ldots,o_t,q_t=i \given \lam)
    \,
    P(o_{t+1}, o_{t+2}, \ldots, o_T \given q_t = i, \lam)
    \\
    &=
    \sum_i P(o_1, \ldots,o_t,o_{t+1}, o_{t+2}, \ldots, o_T,q_t=i \given \lam)
    \\
    &=
    \sum_i P(\vv{o}, q_t=i \given \lam)
    \\
    &= P(\vv{o} \given \lam)
  \end{split}
\end{equation*}
This logic holds for any $t$, so the given sum should be the same for
any $t$.  (The earlier formula for $P(\vv{o} \given \lam)$ was for
the special case $t=T$ since $\beta_T(i)=1$.)  This formula thus
provides a useful debugging test for HMM programs.

\section{Estimate model parameters}

Given $\vv{o}$ find $\lam^* = \argmax_{\lam} P(\vv{o} \given \lam)$.

Not an analytic solution.  Instead, we start with a guess of
$\lam$, typically random, then iterate $\lam$ to a local
maximum, using an EM algorithm.  At each step we ``re-estimate'' a new
$\lam$, called $\hat{\lam}$, which has an increased probability
of generating $\vv{o}$.  (Or if already at a (possibly local)
optimum, the same probability.)

Note: this process is called ``Baum-Welch Re-Estimation.''

Typical stopping rule for this re-estimation loop is:
\begin{align*}
  \text{stop when} \quad
  \log P(\vv{o} \given \hat{\lam})
  - \log P(\vv{o} \given \lam)
  &< \epsilon
  \quad \text{for some small $\epsilon$}
\end{align*}

Note: debugging hint, $P(\vv{o} \given \hat{\lambda}) \geq P(\vv{o}
\given \lambda)$ should always be true.

Definition: $\gamma_t(i) = P(q_t=i \given \vv{o}, \lam)$.  (In
words: the probability of having been in state $i$ at time $t$.)
\begin{equation*}
  \gamma_t(i) =
  \frac{\alpha_t(i) \, \beta_t(i)}{P(\vv{o} \given \lam)}
\end{equation*}

Definition: $\xi_t(i,j) = P(q_t=i, q_{t+1}=j \given \vv{o}, \lam)$.
(In words: the probability of having transitioned from state $i$ to
$j$ at time $t$.)
\begin{equation*}
  \xi_t(i,j) =
  \frac
  {\alpha_t(i) \, a_{ij} \, b_j(o_{t+1}) \, \beta_{t+1}(j)}
  {P(\vv{o} \given \lam)}
\end{equation*}

Note: $\sum_i \gamma_t(i) = 1$ and $\sum_i \sum_j \xi_t(i,j) = 1$.

Note: $\xi$ is written ``xi'' in English, and pronounced ``k'sai.''

We write ``\#'' to abbreviate the phrase ``expected number of times''

\# state $i$ visited: $\dsum_{t=1}^T \gamma_t(i)$

\# transitions from state $i$ to state $j$ is:
$\dsum_{t=1}^{T-1} \xi_t(i,j)$
\begin{align*}
  \hat{\pi}_i &= \frac{\gamma_1(i)}{\dsum_j \gamma_1(j)} = \gamma_1(i)
  \\
  \hat{a}_{ij} &= \frac{\text{\# transitions state $i$ to state
  $j$}}{\text{\# transitions from state $i$}}
  = \frac{\dsum_{t=1}^{T-1}  \xi_t(i,j)}{\dsum_{t=1}^{T-1} \gamma_t(i)}
  \\[2ex]
  \hat{b}_j(k)
  &= \frac{\text{\# in state $j$ and output symbol $k$}}{\text{\# in state $j$}}
  = \frac{\dsum_{t=1}^T [o_t=k] \, \gamma_t(j)}
	 {\dsum_{t=1}^T \gamma_t(j)}
\end{align*}
where we use Knuth notation, $[\emph{boolean\_condition}] = \text{1 or
0}$ depending on whether \emph{boolean\_condition} is true or false.

\subsection{Training on multiple sequences}

The above is for \emph{one} output observable sequence $\vv{o}$.  If
there are multiple such observable output sequences, \emph{i.e.}\ a
training set of them, then the basic variables defined above
($\alpha$, $\beta$, etc) are computed for each of them.  Except for
the re-estimation formulas, which need to sum over them as an
``outer'' sum around the sums shown.

We use a superscript $(p)$ to indicate values computed for observable
sequence $\vv{o}^{(p)}$.  Note that $\lam$ and $N$ and $M$ are
independent of $p$, but $T$ is not since each string in the training
set might be a different length, $T^{(p)} = \dim \vv{o}^{(p)}$.

The update formulas become:
\begin{align*}
  \hat{\pi}_i &= \frac{\dsum_p \gamma^{(p)}_1(i)}{\dsum_p 1}
  \\
  \hat{a}_{ij} &= \frac{\text{\# transitions state $i$ to state $j$}}
		       {\text{\# transitions from state $i$}}
  = \frac{\dsum_p \dsum_{t=1}^{T^{(p)}-1}  \xi^{(p)}_t(i,j)}
	 {\dsum_p \dsum_{t=1}^{T^{(p)}-1} \gamma^{(p)}_t(i)}
  \\[2ex]
  \hat{b}_j(k)
  &= \frac{\text{\# in state $j$ and output symbol $k$}}
	  {\text{\# in state $j$}}
  = \frac{\dsum_p \dsum_{t=1}^{T^{(p)}} [o^{(p)}_t=k] \, \gamma^{(p)}_t(j)}
	 {\dsum_p \dsum_{t=1}^{T^{(p)}} \gamma^{(p)}_t(j)}
\end{align*}

%%% Local Variables:
%%% mode: latex
%%% TeX-master: "example_chapter"
%%% End:
